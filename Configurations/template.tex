

\setcounter{secnumdepth}{6}
\setcounter{tocdepth}{6}

% xelatex 中文套件設定
\usepackage[SlantFont]{xeCJK}
\usepackage{lipsum}
\usepackage{fontspec}
\defaultfontfeatures{AutoFakeBold=2.5,AutoFakeSlant=.2}

\setmainfont[Path=./Fonts/]{times}[        %預設英文字體
	Extension      = .ttf,
	BoldFont       = *bd,
	ItalicFont     = *i,
	BoldItalicFont = *bi
	]
\setmonofont[Path=./Fonts/]{times}
\setsansfont[Path=./Fonts/]{times} 		 %使用目錄底下指定字體

\setCJKmainfont[Path=./Fonts/]{ukai}
\setCJKmonofont[Path=./Fonts/]{edukai-4.0}
\setCJKfamilyfont{edukai-4.0}[Path=./Fonts/]{edukai-4.0}       %教育部楷書

\newcommand{\edukai}{\CJKfamily{edukai-4.0}}

\newcommand\n{\mbox{\qquad}}

\usepackage[LGR,OT1]{fontenc} % Output font encoding for international characters
\usepackage{palatino} % Use the Palatino font by default
\usepackage{indentfirst} \setlength{\parindent}{2em}
\usepackage{xcolor}

\usepackage{lettrine}  % used for chinese bigger capital
\usepackage{algorithm}

\usepackage{enumerate}
\usepackage{multirow}

\frontmatter % Use roman page numbering style (i, ii, iii, iv...) for the pre-content pages
\pagestyle{plain} % Default to the plain heading style until the thesis style is called for the body content
\usepackage{makecell}
\cleardoublepage
\usepackage[autostyle=true]{csquotes} % Required to generate language-dependent quotes in the bibliography

\usepackage{pdfpages}

\usepackage{url}
\usepackage[redeflists]{IEEEtrantools}
\usepackage{indentfirst}

\usepackage{titletoc}
\usepackage{CJKnumb}	% 阿拉伯數字轉中文字

\usepackage{hyperref}

\usepackage{tikz}
\usepackage{eso-pic,picture}
\usepackage{wallpaper}
\usepackage[contents=]{background} %如果沒有給 contest空白會自動加上 draft 浮水印字樣

\newcommand\NTUSTwatermark {
	\backgroundsetup{
		contents={\includegraphics{\watermarkimage}},
		angle=0,
		opacity=0.2,
		hshift = 0mm,
		vshift = 0mm,
		scale=0.5
	}
}

\usepackage{ifthen} % if-else 擴充
\usepackage[subpreambles]{standalone}  % config.tex

\usepackage{pgfplots}
\usepgfplotslibrary{colorbrewer}
\pgfplotsset{compat=newest,compat/show suggested version=false}

\usepackage{threeparttable}

% ----------------------------
%   導入版型設定
% ----------------------------

\usepackage{Templates/abstract_template}
\usepackage{Templates/acknowledgements_template}
\usepackage{Templates/reference_template}
\usepackage{Templates/chapter_template}


% -----------------------------------------------------------------------------------------------


