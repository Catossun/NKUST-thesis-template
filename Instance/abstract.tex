%----------------------------------------------------------------------------------------
%	abstract 摘要
%----------------------------------------------------------------------------------------

\begin{abstract}


測試Nowadays, using minimum sensors to construct a wireless sensor network
such that critical areas in a sensing field can be fully covered has
received much attention recently. However, in reality, the number
of sensors may be limited due to a limited
budget. This motivates us to study the problem of using limited
sensors to construct a wireless sensor network such that the total
weight of the covered critical square grids is maximized, termed the
weighted-critical-square-grid coverage problem, where the critical
grids are weighted by their importance.

In addition, in wireless sensor network,
sensors are responsible for sensing environment and generating
sensed data, and mobile devices are responsible for recharging sensors
and/or collecting sensed data to the sink. Because of the rapid development
of wireless charging technology, sensors can be recharged when
they are within limited charging ranges of mobile devices. In addition,
because sensors’ electric capacity and memory storage are often limited,
sensors must be recharged, and their generated data must be collected
by mobile devices periodically, or the network cannot provide adequate
quality of services. Therefore, the problem of scheduling minimum mobile
devices to periodically recharge and collect data from sensors subject
to the limited charging range, electric capacity, and memory storage,
such that the network lifetime can be guaranteed to be prolonged without
limits, termed the periodic energy replenishment and data collection
problem, is also studied in the thesis.

In other hand, nowadays, wireless sensor network has been widely studied
to combine with many surveillance applications such as Radio frequency
identification (RFID) systems. In RFID systems, a tag can be read by a
reader when the tag is within the reader’s interrogation range. Reader
deployment has received a great deal of attention for providing a
certain service quality. Many studies have addressed deploying/activating
readers such that all tags in a field can be read. However, in practical
environment, tags cannot be read due to collisions. In addition, the number of tags read
by a reader is often limited due to the constraints of processing time
and link layer protocols. This motivated us to study the problem
of activating readers and adjusting their interrogation ranges to cover
maximum tags without collisions subject to the limited number
of tags read by a reader, termed the Reader-Coverage Collision
Avoidance Arrangement (RCCAA) problem.

In this thesis, we study three important problems for wireless sensor
network and wireless sensor network's application. There are deploying
minimum sensors to construct a wireless sensor network such that critical
areas in a sensing field, the energy replenishment and data gathering
in wireless sensor networks whose sensors also can be recharged and
Reader-Coverage Collision Avoidance Arrangement in RFID Networks.
The novel algorithms are proposed corresponding for each part in this
work. In addition, the theoretical analysis of algorithms and some simulations
are also shown to evaluate the performance of our proposed works.

\end{abstract} 