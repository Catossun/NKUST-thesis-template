
\chapter{結論} \label{conclusion_and_future}


\section{研究結論}

本文中介紹了各種優化DFT的架構,並使用Goertzel algorithm的慨念加上我們的想法,提出另一種架構來優化DFT。本文比較各種優化DFT的方法之複雜度,並使用硬體實現各種架構,及使用合成軟體將各種架構的Verilog Code進行電路合成與比較。也將我們優化的DFT使用於SSA大數乘法中,實現兩個 1179648-bits的整數乘法,這個乘法架構可以快速的運算出兩整數相乘之結果。

\section{未來展望}

本論文以有限域下的DFT為基礎,實現另一種DFT的架構。未來可透過不同架構的DFT方法,設計出更簡潔與快速的運算方式。而目前我們將設計之DFT使用到SSA大數乘法中,可以快速計算出兩大整數相乘之結果。而大數乘法可以應用於全同態加密,可以更快的進行運算,將使得全同態加密應用於雲端服務時,雲端服務的安全問題可以得到解決,而且在未來硬體設備提升時,雲端服務可以更安全且快速進行運算。