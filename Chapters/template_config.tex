\chapter{版型設定} \label{ch_tmp_config}

本章包含論文資訊、論文Logo、封面、相關文件引入以及版型微調等。
用於此版型的 config.tex 以及 Configurations 中的 tex 檔案。

\section{論文資訊}

論文資訊主要和論文作者、指導教授、學校等不變的訊息,其設定位於 config.tex 中。
設定中如有碰到 zhtw、tw 或 en 表示這個項目有區分中英文的。

\begin{lstlisting}[language=TeX]
    \def\authortwname{王小明}
    \def\authorenname{Shio-Min Wang}
\end{lstlisting}

目前規範中有一個項目需特別注意,因本校是由三所科技大學合併而成,因此論文需加入原本所屬學校的英文名稱。
下方為原校英文名稱的欄位,請依原校英文名進行修改。

\begin{lstlisting}[language=TeX]
    \def\schoolenoldname{National Kaohsiung University of Applied Sciences}
\end{lstlisting}

\section{初稿與正式版}

給予口試委員的論文為初稿。因此需要封面加入初稿字樣,可透過 config.tex 進行設定。
當設定為 ture 時會產生初稿字樣,設定為 false 表示為正式版初稿字樣將會被隱藏。

\begin{lstlisting}[language=TeX]
    \setboolean{thesisdraft}{true}
    \setboolean{thesisdraft}{false}
\end{lstlisting}

\section{外部檔案匯入與啟用設定}

論文中的封面、書名頁皆可透過 LaTeX 產生,當您已經額外製作封面與書名頁時,可透過外部匯入的方式來取代。
另外博碩士論文授權書、論文口試委員會審定書、論文口試委員會英文審定書、博士論文推薦書皆由外部匯入,論文內無法自動產生。

\subsection*{封面}

封面要使用外部檔案時請將參數 isthesistitleexternal 設定為 true,
並將 externalmaintitle 參數導向至您放置 pdf 檔案的位置。

\begin{lstlisting}[language=TeX]
    \setboolean{isthesistitleexternal}{false}
    \def\externalmaintitle{Externals/maintitle}
\end{lstlisting}

\subsection*{書名頁}
書名頁與封面相同,預設為使用 LaTeX 產生,如需使用外部匯入,請修改 isthesisbooknameexternal 為 true,
並將 externalbooktitle 參數導向至您放置 pdf 檔案的位置。

\begin{lstlisting}[language=TeX]
    \setboolean{isthesisbooknameexternal}{false}
    \def\externalbooktitle{Externals/booktitle}
\end{lstlisting}

\subsection*{授權書與審定書}

授權書、審定書以及英文審定書皆由國家圖書館與學校提供,LaTeX 不提供此版型,因此需額外匯入,
請在匯入時注意修改的規範。載入請設定為 true,不載入請設定為 false。並依照該欄位說明填入指定的 pdf 檔案路徑。

碩博士論文授權書,由國家圖書館發布,依照規定正本應繳回圖書館,
此文件是否需放入論文中尚無定論,端看老師與系辦是否要求,
如需插入本頁文件,應當由您列印文件後簽署,再將簽署好的文件掃描插入此頁中。
\begin{lstlisting}[language=TeX]
    \def\thesispowerofattorney{Externals/powerofattorney.pdf}
    \setboolean{thesisauht}{false}
\end{lstlisting}

當您的論文口試委員審定書使用正本則請忽略此頁,如使用掃描檔案插入則請開啟此頁。
審定書有分中英文,故此將二個項目分開。此文件請由 https://acad.nkust.edu.tw/p/412-1004-2503.php?Lang=zh-tw 進行下載,
下載前建議您先確認本年度是否依然使用此份文件。

\begin{lstlisting}[language=TeX]
    \def\thesisvalidationzhtw{Externals/sign.pdf}
    \setboolean{thesissign_zhtw}{true}
    \def\thesisvalidationen{Externals/sign_en.pdf}
    \setboolean{thesissign_en}{true}
\end{lstlisting}

\subsection*{博士論文推薦書}
碩士不需要使用博士論文推薦書,此項目操作手法與審定書等相同。
\begin{lstlisting}[language=TeX]
    \def\thesisphdrecommand{Externals/recommand.pdf}
    \setboolean{thesisphdrecommand}{false}
\end{lstlisting}

\section{LaTeX 文件啟用與關閉}

\subsection*{誌謝與序言}

誌謝與序言在本專案中被視為相同的文件,載入該文件使用將 thesisacknowledgement 參數設為 true,反之則設為 false。
如您需要修改內容,請由 Instance/acknowledgement.tex 進行編輯。

\begin{lstlisting}[language=TeX]
    \setboolean{thesisacknowledgement}{true}
\end{lstlisting}

\subsection*{目錄列表}

論文中包含了 3 種目錄,文件目錄、圖目錄、表目錄,當您論文沒有使用到圖片或表格時,將會產生多餘的頁數,
因此提供使用者手動屏蔽該頁的功能,載入目錄使用 ture,屏蔽目錄則使用 false。

\begin{lstlisting}[language=TeX]
    \setboolean{thesiscontent}{true}
    \setboolean{thesistable}{true}
    \setboolean{thesisfiguretable}{true}
\end{lstlisting}

\subsection*{附錄}

附錄是論文的附加文件,在本專案的 3 個實驗室中,皆無附加附錄的功能。
因此此欄位預設為關閉狀態,如需啟用請自行在 Configurations/appendice.tex 中加入該附錄內容,如何修改此內容將於下一章節進行說明。

\begin{lstlisting}[language=TeX]
    \setboolean{thesisappendix}{false}
\end{lstlisting}
