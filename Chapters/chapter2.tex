\chapter{內文} \label{ch_content}
參考資料:\cite{latex_basic}, \cite{latex_fbox}
\section{字體}
\subsection{類型}
文字特效, Hello world!      \\
{\LARGE
\textbf{Hello world!, bold face, 粗體} \\
\textit{Hello world!, italic, 斜體} \\
\textsl{Hello world!, slanted, 傾斜} \\
\underline{Hello world!, underline, 底線} \\
}

文字家族, Hello world!      \\
{\LARGE
\textrm{Hello world!, roman, 羅馬} \\
\texttt{Hello world!, typewriter, 等寬} \\
\textsf{Hello world!, sans serif} \\
\textsc{Hello world!, Small Caps} \\
}

\emph{強調}(Emphasized),自動調整字體,使之相對醒目,在不同狀況下有不同效果。

Some of the greatest \emph{discoveries} 
in science 
were made by accident.

\textit{Some of the greatest \emph{discoveries} 
in science 
were made by accident.}

\textbf{Some of the greatest \emph{discoveries} 
in science 
were made by accident.}

\subsection{大小}
文字大小範例, Font Size.\\
{\tiny 文字大小範例, Font Size.}\\
{\scriptsize 文字大小範例, Font Size.}\\
{\footnotesize 文字大小範例, Font Size.}\\
{\small 文字大小範例, Font Size.}\\
{\normalsize 文字大小範例, Font Size.}\\
{\large 文字大小範例, Font Size.}\\
{\Large 文字大小範例, Font Size.}\\
{\LARGE 文字大小範例, Font Size.}\\
{\huge 文字大小範例, Font Size.} \\
{\Huge 文字大小範例, Font Size.}


\section{列表} \label{sec_item}

\begin{description} \item[第1項] 這邊是第1大項
    \item[第2項] 這邊是第2大項
        \begin{description}
            \item[小項] 這邊是第1小項
            \item[小項] 這邊是第2小項
        \end{description}
    \item[第3項] 這邊是第3大項
\end{description}

\begin{itemize}
    \item 這邊是第1大項
    \item 這邊是第2大項
        \begin{itemize}
            \item 這邊是第1小項
            \item 這邊是第2小項
        \end{itemize}
    \item 這邊是第3大項
\end{itemize}

\begin{enumerate}
    \item 這邊是第1大項
    \item 這邊是第2大項
        \begin{enumerate}
            \item[*] 這邊是第1小項
            \item 這邊是第2小項
            \item 這邊是第3小項
        \end{enumerate}
    \item 這邊是第3大項
\end{enumerate}

\newpage

\section{對齊}
\begin{flushleft}
    本段落\\
    向左對齊
\end{flushleft}

\begin{flushright}
    本段落\\
    向右對齊
\end{flushright}

\begin{center}
    本段落\\
    置中對齊
\end{center}

\section{引用}
\subsection{摘寫}\label{ssec_qute}
正文
\begin{quote}
    LATEX 中有三種引用方法:quote、quotation、verse。
    quote:雙邊縮排。
    quotation:雙邊縮排,且句首縮排。
    verse:雙邊縮排,且第二行後縮排。
\end{quote}
正文
\begin{quotation}
    LATEX 中有三種引用方法:quote、quotation、verse。
    quote:雙邊縮排。
    quotation:雙邊縮排,且句首縮排。
    verse:雙邊縮排,且第二行後縮排。
\end{quotation}
正文
\begin{verse}
    LATEX 中有三種引用方法:quote、quotation、verse。
    quote:雙邊縮排。
    quotation:雙邊縮排,且句首縮排。
    verse:雙邊縮排,且第二行後縮排。
\end{verse}

\subsection{交叉引用,引用本文章之內文}
我們寫論文時常需要引用文中section、subsection、figure、table等等內容。此時我們只需要在需要引用的地方加上marker,並使用ref引用標記目標,或使用pageref引用標記處的頁碼。

寫內文可以參考\ref{ch_content}、列表可參考\ref{sec_item}、引文可參考\ref{ssec_qute}、圖表可參考圖\ref{fig_good}

\newpage
\section{盒子}
Latex排板小到一個字母,大到一個段落,都可以視為一個矩型盒子(box),就像html一樣。

mbox:
\mbox{只是一個句子 被組合成盒子 的句子}

fbox:
\fbox{只是一個句子 被組合成盒子 但是有外框 的句子}

makebox/framebox:可加上一些設定的盒子

%語法:[寬][對齊]{內容}
\makebox[\textwidth][c]{只是一個句子 被組合成盒子 的句子}

\framebox[0.5\textwidth][r]{被組合成盒子 但是有外框 的句子 還不夠寬}


\newpage