
\chapter{Introduction} \label{introduction}

中文,測試Due to the growth and advances in networking and electronic hardware
techniques, wireless sensor networks have developed rapidly. In a
wireless sensor network, many small sensors are deployed in a field
to detect the environment and collect sensing data, such as
temperature, humidity, or light data. Each sensor can process,
compute, and transfer data to others in wireless sensor networks.
Many applications based on the wireless sensor networks have been
developed \cite{5705791,5706178,1607984,6421447,7110295,Hammoudi2008343}, such as
medical safety protection, fire and explosion monitors, and
environment monitors.

Nowadays, there have been increasing developments on
the mobility, miniature techniques, size and particularly the diverse
functionalities of hardware. This has made sensor network architecture or the
interactions between network elements more complicated. A
network model is not simply a network model when both hardware
and software of network system has been significantly improved. The designing
and proposition of an optimum solution for network system therefore has
become complicated as well. For example, a wireless sensor network architecture is divided into
layers with different levels of importance rather than having static nature as before.
More complicated network models require new techniques and algorithm to meet real requirements.
Recently, wireless sensor networks are developed rapidly because of
the high achievements of manufacturing technology and information
technology. In the wireless sensor networks, every sensor is able to
sense the environment, compute the data, and communicate with other
sensors. Many applications are developed by the wireless sensor
networks for the monitoring of the environment or the device status.
Because the efficiency of the wireless sensor
network is often related to the sensor deployment, the coverage
problem has thus become a hot research topic, and is studied in this
thesis. In summary, the challenges addressed by this work are:


In addition, because sensors can be deployed to monitor a certain area and can
communicate with each other, a wireless sensor network can thus be
constructed \cite{IEEE1,IEEE2,IEEE3,IEEE4}. Today, wireless sensor
networks have been widely studied for many environment surveillance
applications \cite{ACM14,ACM15,ACM16}. For example,
environmental monitoring wireless sensor networks \cite{IEEE5} can
be used to collect and report environmental humidity and temperature
data. In wireless sensor networks, sensors often need to report the
sensory data back to a certain node, called a sink. Data sensing and
reporting consume most of the sensors' energy. However, the electric
capacity of sensors is limited. Therefore, preventing wireless
sensor networks from collapsing because sensors run out of energy is
a very important issue. In the thesis, the energy
replenishment and data gathering problem in wireless sensor networks addressed whose
sensors can be recharged. Here, these networks are also known as
wireless rechargeable sensor networks (WRSNs).


It is important to note that diversified development of functions of network elements also means more
complicated potential conflicts in network routing and interfacing between
network elements such as RFID networks.
A RFID network is a network that is combined between a RFID system and Wireless sensor network.
To incorporate these functions and avoid these new conflicts,
new network models of high degree of diversification are required.
A RFID system is an automatic identification
system and is composed of tags and readers. In the RFID system,
readers can read the tag's identification and other related
information within the interrogation range \cite{reader1,reader2}.
In addition, tags can be classified as active, passive, and
semi-passive \cite{tag1,tag2}. Passive tags have lower production
cost and are most commonly used in the RFID systems. In this system,
the data obtained by readers can be stored in a sink node. To
provide high-quality service in an RFID system, reader
arrangement/activation has recently become a hot issue \cite{d1,d2}.
In this thesis, reader arrangements also addressed in an RFID system.


\section{Overview of the Proposed Method}

This work addresses the challenges shown previously part for sensor deployment,
the energy replenishment and data gathering and reader arrangement problem
in WSN, WRSN and RFID network, respectively.
In each part, we investigated problem and novel methods are proposed to
address corresponding to the problem. We also evaluate the performance with
our proposed methods in the simulation and provided theoretical analysis
of our proposed method in terms of the correctness, the time complexity,
and the performance guarantee.

In Chapter \ref{chapter1}, we first study the sensor deployment in WSN.
In reality, it is more practical to deploy sensors in
any position in a sensing field. In addition, sensors are often
characterized by various features, such as environmental detection,
intruder detection, and nuclear, biological, chemical (NBC) attack
detection \cite{greenorb, 5235625, intrusion}, due to different monitoring
objectives. The prices of the sensors may be costly such that only
limited sensors can be used for deployment
\cite{5718179,6129271,6095622}. That is, the critical areas in the
field may not be fully covered. Therefore, critical areas must be
weighted by their importance. The more important a critical area,
the higher the weight of the area. For example, in a wilderness
ecological observation network \cite{5719526,COBI:COBI676}, the
nests of animals are more important than their foraging areas. This
motivates us to study a coverage problem, termed the
weighted-critical-square-grid coverage problem. In the problem, the
sensing field is divided into weighted square grids. Although the
number of sensors and the locations of points that are allowed to
deploy sensors are given, the problem is to find a connected
wireless sensor network such that the total weight of the covered
critical grids is maximized.

Then in Chapter \ref{chapter2}, we investigate the Energy Replenishment
and Data Collection problem in WRSNs.
Because of the rapid development of wireless charging technology,
sensors are recharged when they are within limited charging ranges
of mobile devices \cite{INFOCOM6,IEEE7,IEEE12,IEEE17}. In addition,
because the electric capacity and memory storage
\cite{IEEE23,IEEE24} are often limited, sensors must be recharged
and their generated data must be collected by mobile devices
periodically, or the network cannot provide adequate quality of
services. Moreover, because mobile devices with charging capability
are often costly, they have to be used as little as possible.
Therefore, in the thesis, we study the problem of scheduling minimum mobile
devices to periodically recharge and collect data from sensors
subject to the limited charging range, electric capacity, and memory
storage, such that the network lifetime can be guaranteed to be
prolonged without limits, termed the Periodic Energy Replenishment
and Data Collection (PERDC) problem.

Finally, in Chapter \ref{chapter3}, we propose a novel method to
address reader arrangement in RFID network. In RFID networks,
due to the limited energy of readers and the
constraints of processing time and link layer protocols
\cite{related1,related2}, the number of tags read by a reader is
limited. In addition, because a reader's energy consumption is
related to its interrogation range, readers can adjust their
interrogation ranges to save power \cite{WadaHFMO12,4654237}. This
motivates us to study the problem of activating readers and
adjusting their interrogation ranges to cover maximum tags without
collisions subject to the limited number of tags read by a reader,
termed the Reader-Coverage Collision Avoidance Arrangement (RCCAA)
problem.

\section{Contributions}

The original contributions of this work are listed as follow:

\begin{enumerate}

\item We introduce the weighted-critical-square-grid
coverage problem, which is the problem of using limited sensors to
construct a wireless sensor network such that the total weight of
the covered critical square grids is maximized, as will be explained
in Chapter \ref{chapter1}. The problem is shown to be NP-complete.
We propose a relevant approach to the weighted-critical-square-grid
coverage problem.

    \begin{itemize}
      \item  A reduction that transforms
        the weighted-critical-square-grid coverage problem into the
        constrained node-weighted Steiner tree problem was proposed.
      \item Once a solution to the constrained node-weighted Steiner tree problem is
        obtained, the solution can be used to select the points that are
        allowed to deploy sensors for the weighted-critical-square-grid
        coverage problem. We also showed that the constrained node-weighted
        Steiner tree problem is NP-complete.
      \item Three algorithms are proposed for the constrained
        node-weighted Steiner tree problem.
      \item Simulation results demonstrate the performance with our proposed
        methods, including Reduction+GA, Reduction+GBA, and Reduction+PBA, in terms of the total weight of the critical grids covered by
        deployed sensors, where the Reduction+GA, the Reduction+GBA, as will be shown
        in Section \ref{1_Simulation} and the Reduction+PBA denoted the proposed reductions by applying the greedy
        algorithm, the group-based algorithm, and the profit-based
        algorithm, respectively.
    \end{itemize}

\item We introduce the problem of scheduling minimum mobile
devices to periodically recharge and collect data from sensors
subject to the limited charging range, electric capacity, and memory
storage, such that the network lifetime can be guaranteed to be
prolonged without limits, termed the Periodic Energy Replenishment
and Data Collection (PERDC) problem. PERDC problem is shown to be NP-complete.
The appropriate solutions are proposed to PERDC problem.

    \begin{itemize}
      \item  Three algorithms, including the grid-cell-based algorithm (GSBA), the
        dominating-set-based algorithm (DSBA), and the
        circle-intersection-based algorithm (CIBA), are proposed to find a
        set of anchor points. Based on the generated anchor points, the
        mobile device scheduling algorithm (MDSA) was proposed to schedule
        minimum mobile devices for energy replenishment and data collection, as will be shown
        in Section \ref{2_section:method}.
      \item Theoretical analysis and simulation results demonstrate that our proposed methods
      provides better performance than others.
    \end{itemize}

\item A general problem of activating readers and adjusting their interrogation ranges to cover
        maximum tags without collisions subject to the limited number of tags read by a reader, termed the Reader-
        Coverage Collision Avoidance Arrangement (RCCAA) problem, is first discussed and proposed in this work. In addition, the difficulty of the RCCAA is provided.

    \begin{itemize}
      \item By considering the limited number of tags read by a reader, a novel method, termed the Maximum-Weight-Independent-Set-Based Algorithm (MWISBA), is proposed for the RCCAA problem.
      \item The correctness and the time complexity of the MWISBA is provided by theoretical analysis.
      \item The performance ratio of the MWISBA is shown to be at least $1/\theta$.
      \item Simulation results demonstrate that the MWISBA provides better performance than heuristics.
    \end{itemize}

\end{enumerate}
\section{gioi thieu ban than}
\subsection{gioi thieu ve bo}
xin gioi thieufsfsfsf
\subsection{gioi thieu ve me}
day la me toi

\section{Thesis Structure}

The thesis is laid out as follows:

\begin{itemize}
  \item In Chapter \ref{background}, we review existing methods and introduce in more detail related works that
    inspired the presented work, and methods which we will used to construct our proposed methods.
  \item In Chapter \ref{chapter1}, we introduce the weighted-critical-square-grid
    coverage problem and its hardness, and we provide the appropriate approach and evaluations of the performance of our proposed algorithms in each method, receptively.
  \item In Chapter \ref{chapter2}, we propose efficient algorithms to address Periodically Energy charging and Data Collection problem with minimum mobile devices in WRSNs. The performance of our proposed algorithms are also shown through theoretical analysis and simulation results in this Chapter.
  \item In Chapter \ref{chapter3}, we first review the Reader Coverage Collision in RFID networks then we propose
    a novel method to avoid collision for Reader in coverage tag process. Finally, theoretical analysis, the performance ratio and simulation results of proposed an algorithm are also shown.
  \item Chapter \ref{conclusion_and_future} concludes this thesis and discusses possible extensions of this work.
\end{itemize}





